%% @Author: Ines Abdeljaoued Tej
%  @Date:   2018-06
%% @Class:  PFE de l'ESSAI - Universite de Carthage, Tunisie.


\markboth{\MakeUppercase{Introduction}}{}%
\addcontentsline{toc}{chapter}{Introduction}%

%Welcome to \Ac{ESSAI}. ~\\
%Again, welcome to \Ac{ESSAI}. ~\\
%Your introduction goes here. ~\\

Voici une référence à l'image de la Figure \ref{fig:test} page \pageref{fig:test} et une autre vers la partie \ref{chap:2} page \pageref{chap:2}.
On peut citer un livre\, \cite{caillois1} et on précise les détails à la fin du rapport dans la partie références.
Voici une note\,\footnote{Texte de bas de page} de bas de page\footnote{J'ai bien dit bas de page}. Nous pouvons également citer l'Algorithme \ref{algo1}, la Définition \ref{def1}, le Théorème \ref{theo1} ou l'Exemple \ref{exo1}...\\

Le document est déatillé comme suit : le chapitre \ref{chap:chapterone} introduit le cadre général de ce travail. Il s'agit de présenter l'entreprise d'accueil et de détailler la problématique. Le chapitre \ref{chap:2} introduit les données ainsi que les modèles choisies. Le chapitre \ref{chap:3} donne les principaux résultats et la comparaison entre divers modèles (courbe de ROC, indice de Gini). Nous clôturons ce travail par une brève conclusion résumant le travail accompli ainsi que des perspectives qui pourraient enrichir ce travail.  


